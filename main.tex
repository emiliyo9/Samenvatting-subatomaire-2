\documentclass{article}
\usepackage[utf8]{inputenc}
\usepackage{subfiles}
\usepackage{hyperref}
\usepackage{multirow}
\usepackage{graphicx}
\usepackage{amsmath}
\usepackage{tikz-feynman}
\usepackage[utf8]{inputenc}

\title{Samenvatting: subatomaire fysica 2}
\author{Emile Segers}
\date{Year 2019-2020}

\begin{document}

\maketitle

\begin{abstract}
    Dit is een samenvatting gebasseerd op de lessen subatomaire fysica 2 2020-2021. Dit is geen vervanging voor de cursus gegeven in dit vak. Het doel van deze samenvatting is een studiehulp te zijn bij de lessen van Dirk Ryckbosch. Gebruik dit dan ook enkel als hulp.\\
    De schrijver van deze samenvatting is niet verantwoordelijk voor het maken van fouten op een examen of ergens anders. Indien je fouten vindt kan je me altijd contacteren op dit e-mailaddres \href{mailto:emile.segers8@gmail.com}{emile.segers8@gmail.com}.\\
    De voertaal van het examen subatomaire fysica 2 in 2020-2021 is Nederlands. Deze samenvatting zal dus ook grotendeels in het Nederlands geschreven worden.
\end{abstract}

\tableofcontents

%\subfile{introduction_and_review/main.tex}

%\subfile{quantum_numbers/main.tex}

%\subfile{feynman_diagrams_processes_corrections/main.tex}

\subfile{DIS_nucleon_structuur_pdf/main.tex}

\end{document}
