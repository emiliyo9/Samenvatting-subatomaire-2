\documentclass{article}
\usepackage[utf8]{inputenc}
\usepackage{subfiles}
\usepackage{hyperref}
\usepackage{multirow}
\usepackage{graphicx}
\usepackage{amsmath}
\usepackage{tikz-feynman}
\usepackage{subfig}
\usepackage{wrapfig}
\usepackage{centernot}
\usepackage{color,soul}
\usepackage[dutch]{babel}
\hyphenation{ef-fi-ciën-tie}

\title{Samenvatting: subatomaire fysica 2}
\author{Emile Segers}
\date{Jaar 2020-2021}

\begin{document}

\maketitle

\begin{abstract}
    Dit is een samenvatting gebaseerd op de lessen subatomaire fysica 2 2020-2021. Dit is geen vervanging voor de cursus gegeven in dit vak. Het doel van deze samenvatting is een studiehulp te zijn bij de lessen van professor Ryckbosch. Gebruik dit dan ook enkel als hulp.\\
    De schrijver van deze samenvatting is niet verantwoordelijk voor het maken van fouten op een examen of ergens anders. Indien je fouten vindt, kan je me altijd contacteren op dit e-mailadres \href{mailto:emile.segers8@gmail.com}{emile.segers8@gmail.com}.\\
    De voertaal van het examen subatomaire fysica 2 in 2020-2021 is Nederlands. Deze samenvatting zal dus ook grotendeels in het Nederlands geschreven worden.\\
    {\color{red} Disclaimer:} niet alle zinnen zijn altijd even correct geschreven. Indien je deze kleine fouten wilt aanpassen en niet altijd een berichtje wilt sturen, kan je altijd aanpasrechten vragen in overleaf om de tekst zelf aan te passen.\\
    Indien je hier zelf verder aan zou willen werken, kan je ook altijd het \href{https://github.com/emiliyo9/Samenvatting-subatomaire-2}{git project} of het \href{https://www.overleaf.com/read/nzjkpzmjzjpd}{overleaf project} te forken.

\end{abstract}

\tableofcontents

\subfile{introduction_and_review/main.tex}

\subfile{quantum_numbers/main.tex}

\subfile{feynman_diagrams_processes_corrections/main.tex}

\subfile{DIS_nucleon_structuur_pdf/main.tex}

\subfile{QCD/main.tex}

\subfile{parity_violation/main.tex}

\subfile{elektroweak_precision_tests/main.tex}

\subfile{higgs_boson/main.tex}

\subfile{meson_mixing_and_oscillations/main.tex}

\subfile{cp_violation/main.tex}

\subfile{neutrinos/main.tex}

\subfile{physics_beyond_the_standard_model/main.tex}

\end{document}
